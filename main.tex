\documentclass{article}


\usepackage[mitschrift,aufgaben, skript,chinesisch]{kern}
\geometry{
    letterpaper, 
    twoside=true, 
    bindingoffset=1cm, 
    left=3.5cm, 
    right=3.5cm, 
    top=3.5cm, 
    bottom=3.5cm
}

\ihead{\textbf{Integrierter Kurs IV}\\\textit{Theoretische Physik II}\\\texttt{Skript}}
\chead{\textit{Tom Folgmann}}

\title{Integrierter Kurs IV}
\author{Theoretische Physik II\\Tom Folgmann}

\begin{document}
    \maketitle
    \noindent\anm{Das Passwort für die offiziellen Kursfolien ist \enquote{2023ik4}.}
    \tableofcontents

    \section{Einleitung und Wellenfunktion}
        \subfile{Notes/lec1.tex}
        \subfile{Notes/lec2.tex}
        \subfile{Notes/lec3.tex}
        \subfile{Notes/lec4.tex}
        \subfile{Notes/lec5.tex}
        \subfile{Notes/lec6.tex}
        \subfile{Notes/lec7.tex}
        \subfile{Notes/lec8.tex}
        
    \section{Lösen von Bewegungsgleichungen der Schrödingergleichung}
        \subfile{Notes/lec9.tex}
        \subfile{Notes/lec10.tex}
        \subfile{Notes/lec11.tex}
        \subfile{Notes/lec12.tex}

    \section{Drehimpuls und Bewegung im Zentralfeld}
        \subfile{Notes/lec13.tex}

    % Hier waren Ferien:
        \subfile{Notes/lec14.tex}
        \subfile{Notes/lec15.tex}

    \section{Mathematische Grundlagen}
        \subfile{Notes/lec16.tex}
        \subfile{Notes/lec17.tex}
        \subfile{Notes/lec18.tex}
        \subfile{Notes/lec19.tex}
        \subfile{Notes/lec20.tex}
        \subfile{Notes/lec21.tex}
        \subfile{Notes/lec22.tex}
        
    % Störungstheorie:
        \subfile{Notes/lec23.tex}
        \subfile{Notes/lec24.tex}

    \newpage
    \subfile{Anhang/Literatur.tex}
\end{document}