\documentclass{subfiles}

\begin{document}
    \marginnote{\textbf{\textit{VL 6}}\\05.05.2023, 11:45}

    \subsection{Operatoren}
        Zunächt sei ein dringender Verweis zur \emph{Funktionalanalysis I\&II} und \emph{Mathematische Grundlagen der Quantenmechanik} gegeben. \\

        Wir stellen uns die Frage, ob $<p>$ oder $<g(p)>$ direkt aus $\psi(t,r(t))$ bei fixiertem $t\in\R$ berechnet werden kann, ohne die \emph{Fouriertransformation} zu verwenden. Zunächst gilt 
        \begin{align*}
            <p>:= E_{P(\mcF\psi),p}(t) &= \frac{1}{(2\pi)^3}\cdot\lint{p\cdot\abs{\mcF\psi(t,p)}^2}{p}{\R^3}, 
        \end{align*}
        wobei wir den \emph{komplexen Betrag} der Fouriertransformierten von $\psi$ verwenden. Es gilt weiter nach Definition
        \[\frac{1}{(2\pi)^3}\cdot\lint{p\cdot\lint{\lint{\overline{\psi(R)}\cdot\psi(r)\cdot\exp(-\frac{\cmath\cdot p\cdot (r-R)}{\hbar})}{R}{\R^3}}{r}{\R^3}}{p}{\R^3},\]
        wobei wir durch Umsortieren der Integrale 
        \[\frac{1}{(2\pi)^3}\cdot\lint{\nbra{\lint{\nsqbra{\lint{\overline{\psi(R)}\cdot\psi(r)\cdot \cmath\hbar\cdot D_h(\exp\circ g_R)(r)}{R}{\R^3}}}{r}{\R^3}}}{p}{\R^3}\]
        mit $g_R:=(-\cmath\cdot p\cdot(r-R))_{r\in\R^3}$ und $h\in\R^3$. Mit partieller Integration $\mint{f'(x)\cdot g(x)}{\mu(dx)}{[a,b]} = [f(x)\cdot g(x)]_a^b-\mint{f(x)\cdot g'(x)}{\mu(dx)}{[a,b]}$ folgt
        \[-\lint{\lint{\overline{\psi(R)}\cdot\cmath\hbar\cdot D_h\psi(r)}{R}{\R^3}}{r}{\R^3}\cdot\ubra{\lint{\frac{1}{(2\pi)^3}\cdot\exp(\frac{-\cmath p\cdot(r-R)}{\hbar})}{p}{\R^3}}{=\mint{R}{\delta_r}{\R^3}}.\]
        Mit $\mint{R}{\delta_r}{\R^3} = 1$ für $r=R$ folgt 
        \[\lint{\overline{\psi(r)\cdot(-\cmath\hbar D_h)\psi(r)}}{r}{\R^3} =: <P>,\]
        wobei $<\cdot>$ den Erwartungswert des $\cdot$-es ist, welcher in diesem Fall der \emph{Operator} P ist. 

        \begin{Aufgabe}
            \nr{} Verifiziere $\mcF(1)_{x\in\R^3} = \mint{x}{\delta_{0_\R^3}}{\R^3}$. Schreibe hierzu die Fouriertransformation $\mcF(1)_{x\in\R^3}$ aus. Was ergibt $\lint{f(x)\cdot\mcF(1)(x)}{x}{\R^3}$? 
        \end{Aufgabe}
        \subsubsection*{Zusammenfassen}
            Für einen linearen stetigen Operator $T\in L_S(\mcH)$ auf dem Hilbertraum $\mcH$ gilt für den Erwartungswert 
            \[<T> := \lint{\overline{\psi(r)}\cdot (T\circ\psi)(r)}{r}{\mcH}.\]

            \begin{Aufgabe}
                \nr{} Zeige der vorigen Rechnung folgend die Aussage $<g(p)> = <g(P)>$ für $p\in\R^3$ und $P\in L_S(\R^3)$, indem man für analytische $g$ eine Potenzreihenentwicklung durchführt. 
            \end{Aufgabe}

        Wenn $x\in\R^3$ ein Vektor der Form $[3]\to\R$ ist, dann ist der zugeörige Operator $X$ eine Abbildung aus dem Definitionsbereich von $x$ in den Raum $L_S(\R)$ der stetigen linearen Operatoren auf $R$ gemäß $X:[3]\to L_S(\R)$. 

        \subsubsection*{Operatoren der Quantenphysik}
            In der Quantenmechanik beschreiben wir Observable nun durch Identifikation mit Operatoren:
            \begin{table}[H]
                \centering
                \begin{tabular}{c|c}
                    Messgröße & Operator \\
                    \hline
                    Energie & \enquote{$\hat H = H(t,(r(t),\hat p(t)))$} \\
                    Impuls & \enquote{$\hat p = -\cmath\hbar D_h$} \\
                    Ort & \enquote{$\hat r = r(t)$}
                \end{tabular}
            \end{table}
            Wie in der Physik üblich handelt es sich hier allerdings nur um \emph{Sprechweisen}, welche an die mathematischen Hintergründe im physikalisch ausreichenden Sinne \emph{erinnern}. \\

            Die sogenannten \emph{Eigenzustände} sind Lösungen des verallgemeinerten Eigenwertproblemes $T(\psi) = \lambda\cdot\psi$ zu dem verallgemeinerten Eigenwert $\lambda\in\C$ und einem Operator $T\in L_S(\mcH)$ über dem komplexen Hilbertraum $\mcH$. Ist $\psi$ ein solcher Eigenzustand von $T$, so ist der Erwartungswert 
            \begin{multline*}
                E_{T(\psi),}(t) = \lint{\overline{\psi(t,r)}\cdot (T\circ\psi)(t,r)}{r}{\R^3} \\
                = \lint{\overline{\psi(t,r)}\cdot \lambda\cdot\psi(t,r)}{r}{\R^3} = \lambda\cdot E_{P(\psi),r}(t)(\R^3) = \lambda.
            \end{multline*}
            
            \begin{Aufgabe}
                \nr{} Zeige die Aussage $E_{T(\psi),x}(t)(\R^3) = \lambda^2$. 

                \nr{} Zeige die \emph{Varianz} des Operators $T\in L_S(\H)$ mit $\psi\in\H$ als Eigenzustand zu $\lambda\in\C$. Erhalte $\langle\text{var}(T^2)\rangle = \lambda^2-\lambda^2 = 0$. Benutze hierzu das Ergebnis $E_{T^2(\psi),x}(t) = \lambda^2$. 
            \end{Aufgabe}


\end{document}