\documentclass{subfiles}

\begin{document}
    \marginnote{\textit{\textbf{VL 13}}, 24.05.2023, 08:15}
    Gleich zu Beginn betonen wir, daß wir im Folgenden die Absicht haben, in \emph{drei Dimensionen} zu arbeiten. Wir betrachten also nun $\mcL^2(\R^3)$ und entsprechende Operatoren. Die optische Form der Eigenwertgleichung bleibt jedoch erhalten: $H(\psi) = \lambda\cdot\psi$. Für ein Elektron-Proton Paar betrachten wir das Coulombpotential mit
    \[V := \fdef{-\frac{e^2}{4\pi\cdot\epsilon_0}\cdot\frac{1}{\dabs{x}{3}}}{x\in\R^3}.\]
    \subsection{Orientierungs- und Drehimpulsalgebra}
    Wir betrachten hier die Drehgruppe $\text{SO}_3$. Diese besteht dabei aus allen invertierbaren Funktionen $f$ aus $\R^3$ nach $\R^3$, welche in der Verkettung mit der transponierten die Identität ergeben, also in Formeln
    \[
        \text{SO}_d:=\nset{A\in\R^{d\times d}:A\cdot A^T = A^T\cdot A = I_n,\;\det(A)=1}.
    \]
    Für eine bessere Ordnung beschreiben wir die Matrix $A$, welche einen Vektor $x\in\R^3$ um die Achse $a\in\R^3$ genau um den Winkel $\varphi\in\R$ dreht, als
    \[
        \mcD_{\varphi,a}:=\Eintrag{\nset{A\in\text{SO}_3:\frac{\scpr{x}{Ax}}{\dabs{x}{2}\cdot\dabs{Ax}{2}} = \cos(\varphi),\;\textit{rotiert um }a}}.
    \]
    Dies ist äquivalent zu der Schreibweise
    \[\mcD_{\varphi,a}x = x+\frac{a}{\dabs{a}{2}}\times x.\]
    \begin{Aufgabe}
        \nr{} Formuliere zunächst die Definition von $\mcD_{\varphi,a}$ aus. Zeige dann, dass $\mcD_{\varphi,a}$ eine Drehmatrix ist. Betrachte dann die alternative Schreibweise und verifiziere, dass diese äquivalent ist.
    \end{Aufgabe}

    \subsubsection*{Der Drehimpulsoperator}
        Um fortschreitend in mehreren Dimensionen über Operatoren sprechen zu können, benötigen wir zunächst die Definition des Operatortupels.
        \begin{mdef}{Das Operatortupel}
            Als Operatortupel definieren wir eine Abbildung $[N]\to(\mcH\to\mcH)$. Die Verkettung zweier Operatortupel ist dabei definiert als punktweise Verkettung $T^2 = (T_i\circ T_i)_{i\in\Def(T)}$ und die Auswertung in $x\in\mcH$ als $T(x) := \sum_{i\in\Def(T)}T_i(x)$.  
        \end{mdef}
        Um die Drehung in der Anwendung zu sehen, betrachten wir den Drehimpulsoperator $n\mapsto L_n\in L_S(\R)$. Diesen definieren wir druch $L:=Q\times P$, wobei wir mit $Q$ und $P$ Orts- und Impulsoperatortupel meinen und das Kreuzprodukt wie in $\R^3$ definieren. Wendet man die Drehung auf ein $\psi\in\mcL^2(\R)$ an, gilt
        \[
            \mcD_{\varphi,a}\psi(x) = \psi(x) - \frac{a}{\dabs{a}{2}}\times Q\cdot D_{1_\R^3}\psi(x) = \psi(x) - \frac{\cmath}{\hbar}\cdot\frac{a}{\dabs{a}{}}\cdot(Q\times P)\psi(x),
        \]
        wodurch wir einen neuen Zusammenhang
        \[\mcD_{\varphi,a} = I_3 - \frac{\cmath}{\hbar}\cdot a\cdot L\]
        mit unserem definierten Drehimpuls $L$ vorfinden. 
        \begin{mdef}{Der Drehimpulsoperator}
            Ein Operatortupel $n\mapsto J_n\in L_S(\R)$ nennen wir \emph{Drehimpulsoperator(tupel)}, wenn $[L_i,L_j] = \cmath\hbar\cdot\sum_{k\in[3]}\epsilon_{ijk}\cdot L_k$ für $i\neq j\in\Def(J)$ gilt. 

            Speziell im physikalischen Sinne ist $L:=Q\times P$ in drei Dimensionen mit dem Erwartungswert $\sum_{i\in[3]}\langle L\rangle_\psi$ und den der Hamliton-Kommutatoreigenschaft $[L,H] = 0$. 
            
            In der Verkettung mit sich selbst gilt die Eigenwertgleichung $L^2(\psi) = \hbar^2\lambda\cdot\psi$.
        \end{mdef}
        \begin{Aufgabe}
            \nr{} Zeige, daß $L$ ein selbstadjungierter Operator ist. 

            \nr{} Zeige durch die Symmetrie des Zentralpotentials $V:\R^3\to\R$ die Kommutatoreigenschaft $[\mcD_{\varphi,a},H] = 0$, wobei $H$ der Hamiltonoperator ist. 

            \nr{} Beschäftige dich mit der Komponentenschreibweise der Operatoren $H$ und $L$. Zeige $[L_i,L_j]\neq 0$ für $i\neq j$. Was bedeutet hier die Notation?
        \end{Aufgabe}
        Aus der Aufgabe und der Schreibweise für $\mcD_{\varphi,a}$ folgern wir weiter $[L,H] = 0$. 
        Mit der Aufgabe folgern wir nun eine allgemeinere: Für $(i,j)\in[3]^2$ gilt
        \[
            [L_i,L_j] = \cmath\hbar\cdot \sum_{k=1}^3\epsilon_{ijk}\cdot L_k \implies L\times L = \cmath\hbar\cdot L,
        \]
        wobei $\epsilon$ als Rechenhilfe unter dem Namen \textit{Levi-Civita-Symbol} bekannt ist. 
        \begin{Aufgabe}
            \nr{} Folgere nun als letzte Eigenschaft aus den bisherigen $[L^2,H] = 0$ und $[L^2,L] = 0$.
        \end{Aufgabe}
        Betrachten wir den Erwartungswert $\langle L^2\rangle_\psi$, so finden wir nach Definition $\langle T\rangle_\psi := \sum_{i\in\Def(T)}\langle T\rangle_\psi$ für ein Operatortupel den Wert
        \[
            \dabs{L^2(\psi)}{} = \scpr{L(\psi)}{L(\psi)} = \scpr{\psi}{L^2(\psi)} = \sum_{i\in[3]}\scpr{L_i(\psi)}{L_i(\psi)}\geq 0.
        \]
        Somit sind durch Selbstadjungiertheit und positiv Definitheit des Skalarproduktes die Eigenwerte von $L^2$ im reellen und größer Null: $L^2\psi = \hbar^2\lambda\psi$. 
        \begin{mdef}{Der Leiteroperator}
            Als Leiteroperator $L_\pm$ bezeichnen wir die Operatoren $L_\pm := L_1\pm\cmath\cdot L_2$. Zur Adjunktion gilt $L_\pm^* = L_\mp$.
        \end{mdef}
        \noindent Für die Leiteroperatoren gelten eine Reihe von Eigenschaften:
        \begin{alignat*}{4}
            &\scpr{\psi}{L_{pm}(\psi)} &&= \scpr{L_\mp(\psi)}{\psi} &&\qquad [L_3,L_\pm] &&= \pm\hbar\cdot L_\pm\\
            &[L_+,L_-] &&= 2\hbar\cdot L_3 &&\qquad [L^2,L_\pm] &&= 0\\
            &L_+\circ L_- &&= L_1^2 + L_2^2 + \hbar\cdot L_3 &&\qquad L^2 &&= L_-\circ L_+ + \hbar L_3 + L_3^2.
        \end{alignat*}
        
        \begin{Aufgabe}
            \nr{} Zeige diese Kommutatoreigenschaften des Leiteroperators $L_\pm$ in Verbindung mit $L^2$ und $L_3$. Zeige hierfür auch $L_\pm^* = L_\mp$. 
        \end{Aufgabe}
        Um den Zusammenhang zwischen $L$ und $L_\pm$ zu erkennen, behaupten wir nun $L_\pm(\psi)$ ist Eigenvektor von $L_z$ mit Eigenwert $\hbar$. Wir finden
        \[(L_z\circ L_\pm)(\psi) = (L_\pm\circ L_z)(\psi) \pm\hbar L_\pm(\psi) = h\cdot (m\pm 1)\cdot L_\pm(\psi),\]
        wobei $L_z(\psi) = h\cdot m\cdot \psi$ gelte. 

        \begin{Aufgabe}
            \nr{} Zeige $\psi$ ist Eigenvektor von $L^2$ mit Eigenwert $\hbar^2\cdot\lambda$. Zeige dann auch $L_\pm(\psi)$ ist Eigenvektor mit selbem Eigenwert. 
        \end{Aufgabe}

        Ist $\psi$ ein normierbarer Eigenvektor zu $L^2$, dann ist $L_\pm(\psi)$ ebenfalls normierbar:
        \[
            \dabs{L_z(\psi)}{}^2 = \scpr{\psi}{(L_\mp\circ L_\pm)(\psi)} = \hbar^2\cdot (\lambda-m^2\mp m)\cdot\scpr{\psi}{\psi} \geq 0.
        \]
        Wir können also sehen, daß nicht unbedingt jeder Eigenvektor von $L^2$ sich zur Normierbarkeit eignet. 

        \begin{Aufgabe}
            \nr{} Betrachte die Eigenwertgleichungen (i) $L^2\psi_{l,m} = \hbar^2\cdot l\cdot(l + 1)\cdot\psi_{l,m}$, (ii) $L_z(\psi_{l,m}) = \hbar\cdot m\cdot \psi_{l,m}$ und (iii) $L_\pm(\psi_{l,m}) = \hbar\cdot\sqrt{l\cdot (l+1) - m\cdot(m\pm 1)}\cdot\psi_{l,m}$. Was sagen die Zahlen $l,m\in\N$ aus?
        \end{Aufgabe}

    \subsection{Ortsdarstellung und Kugelflächenfunktionen}
        Wir suchen nun einen konkreten Ausdruck für die Eigenvektoren $\psi_{l,m}\in\mcL^2(\R^3)$ der Eigenwertgleichung $L^2(\psi_{l,m}) = \lambda\cdot \psi_{l,m}$ mit Drehimpulsquantenzahl $l\in[n-1]$ und $m_l\in D_l\cap [-l,l]$, wobei $D_f:=\{-l + n:n\in\N_0\}$. Nach der Definition von $L$ steht in ausgeschriebener Form zunächst mit $\nabla$ als Differntialoperator folgender Ausdruck:
        \begin{align*}
            L^2(\psi_{l,m}) &= -\hbar\cdot(x\times \nabla)^2(\psi_{l,m}) = \hbar^2\cdot l\cdot (l+1)\cdot \psi_{l,m}\\
            L_3(\psi_{l,m}) &= -\cmath\hbar\cdot (x\times \nabla)_3^2(\psi_{l,m}) = \hbar\cdot m\cdot \psi_{l,m}.
        \end{align*}
        Unter der Transformation in Kugelkoordinaten $f_K:x\mapsto (r(x),\vartheta(x),\varphi(x))$ mit $x\in\R$ und $\vartheta:\R\to (0,\pi)$ und $\varphi:\R\to (-\pi,\pi)$ als Winkelzuordnungen und \emph{Trennung der Variablen} der Form  
        \[
            \psi_{l,m}(x) = (R_{l,m}\circ (f_K)_1^*)(x)\cdot (Y_{l,m}\circ ((f_K)_2^*,(f_K)_3^*))(x)
        \]
        und $Y_{l,m}(\vartheta_x,\varphi_x) = \Phi_{l,m}(\varphi_x)\cdot\Theta_{l,m}(\vartheta_x)$ kann man zunächst $\Phi_{l,m}(\varphi_x) = \exp(\cmath\cdot m\cdot\varphi_x)$ setzen. Es ist dann zu Lösen eine Gleichung der Form 
        \[
            \nbra{\dv{z}(1-z^2)\cdot\dv{z} + \Bbra{l\cdot(l+1) - \frac{m^2}{1-z^2}}}(\Theta_{l,m}(\vartheta)) = 0.
        \]
        Wir nennen die Gleichung auch \emph{verallgemeinerte Legendre Gleichung}, deren Lösung die \emph{verallgemeinerten Legendre Polynome} sind. Diese sind von der Form
        \[
            \Theta_{l,m}(\vartheta) = \sum_{k=0}^l a_k\cdot P_{k,m}(\cos(\vartheta)),
        \]
        wobei $P_k^m$ die \emph{asoziierten Legendre Polynome} sind. Diese sind wiederum von der Form
        \[
            P_{k,m}(z) = (-1)^m\cdot(1-z^2)^{\frac{m}{2}}\cdot\dv{z}^m\cdot P_k(z).
        \]
        Für $Y$ finden wir eine Gleichung der Form
        \[
            Y_{l,m}(\vartheta,\varphi) = \sqrt{\frac{2l + 1}{4\pi}\cdot\frac{(l-m)!}{(l+m)!}}\cdot P_e(\cos(\vartheta))\cdot\exp(-\cmath m\cdot\varphi),
        \]
        die sogenannte \emph{Kugelflächenfunktion}. 
        \begin{mcor}{Satz von Kugelflächenfunktionen in Polardarstellung des Drehoperators}
            Die Polartransformation der Drehimpulsoperatorverkettung $L^2$ mittels $f_K$ liefert für eine Wellenfunktion $\psi_{l,m}\in H^2(\R)$ aus der Eigenwertgleichung $L^2(\psi_{l,m}) = \lambda\cdot\psi_{l,m}$ die Zerlegungsmöglichkeit $\psi_{l,m} = R_{l,m}\circ (f_K)_1^* \cdot Y_{l,m}\circ\bbra{(f_K)_2^*,(f_K)_3^*}$ in die sogenannte \emph{Kugelflächenfunktion} $Y_{l,m}$ und die \emph{Radialfunktion} $R_{l,m}$. 

            Die Kugelflächenfunktionen $Y_{l,m}$ sind dabei für erlaubte $l,m$ \emph{orthonormal} und \emph{vollständig}, sowie erfüllen die \emph{Parität} $Y_{l,m}(-x) = (-1)^l\cdot Y_{l,m}(x)$ und die \emph{komplexe Konjugation} $\overline{Y_{l,m}(x)} = (-1)^m\cdot Y_{l,-m}(x)$.
        \end{mcor}
        \begin{Aufgabe}
            \nr{} Zeige die Orthogonalität $Y_{l,m}$ mit $Y_{l',m'}$ für erlaubte $l,m$ und $l',m'$.

            \nr{} Zeige die Vollständigkeit der Kugelflächenfunktionen nach Definition 
            \[
                \sum_{l=0}^\infty\sum_{m=-l}^l Y_{l,m}(x)\cdot\overline{Y_{l,m}(x')} = \delta(x-x').
            \]

            \nr{} Zeige die Paritätseigenschaft $Y_{l,m}(-x) = (-1)^l\cdot Y_{l,m}(x)$.

            \nr{} Zeige die Eigenschaft der komplexen Konjugation $\overline{Y_{l,m}(x)} = (-1)^m\cdot Y_{l,-m}(x)$.
        \end{Aufgabe}

    

    

\end{document}