\documentclass{subfiles}

\begin{document}
    \marginnote{\textit{\textbf{VL 13}}, 24.05.2023, 08:15}
    Gleich zu Beginn betonen wir, daß wir im Folgenden die Absicht haben, in \emph{drei Dimensionen} zu arbeiten. Wir betrachten also nun $\mcL^2(\R^3)$ und entsprechende Operatoren. Die optische Form der Eigenwertgleichung bleibt jedoch erhalten: $H(\psi) = \lambda\cdot\psi$. Für ein Elektron-Proton Paar betrachten wir das Coulombpotential mit
    \[V := \fdef{-\frac{e^2}{4\pi\cdot\epsilon_0}\cdot\frac{1}{\dabs{x}{3}}}{x\in\R^3}.\]
    \subsection{Orientierungs- und Drehimpulsalgebra}
    Wir betrachten hier die Drehgruppe $\text{SO}_3$. Diese besteht dabei aus allen invertierbaren Funktionen $f$ aus $\R^3$ nach $\R^3$, welche in der Verkettung mit der transponierten die Identität ergeben, also in Formeln
    \[\text{SO}_d:=\nset{A\in\R^{d\times d}:A\circledast A^T = A^T\circledast A = I_n,\;\det(A)=1}.\]
    Für eine bessere Ordnung beschreiben wir die Matrix $A$, welche einen Vektor $x\in\R^3$ um die Achse $a\in\R^3$ genau um den Winkel $\varphi\in\R$ dreht, als
    \[\mcD_{\varphi,a}:=\Eintrag{\nset{A\in\text{SO}_3:\frac{\scpr{x}{Ax}}{\dabs{x}{2}\cdot\dabs{Ax}{2}} = \cos(\varphi),\;\textit{rotiert um }a}}.\]
    Dies ist äquivalent zu der Schreibweise
    \[\mcD_{\varphi,a}x = x+\frac{a}{\dabs{a}{2}}\times x.\]
    \begin{Aufgabe}
        \nr{} Formuliere zunächst die Definition von $\mcD_{\varphi,a}$ aus. Zeige dann, dass $\mcD_{\varphi,a}$ eine Drehmatrix ist. Betrachte dann die alternative Schreibweise und verifiziere, dass diese äquivalent ist.
    \end{Aufgabe}

    \subsubsection*{Der Drehimpulsoperator}
        Um die Drehung in der Anwendung zu sehen, betrachten wir den Drehimpulsoperator $L\in L_S(\R^3)$. Diesen definieren wir druch $L:=r\times p$, wobei wir mit $r$ und $p$ den Orts- und Impulsoperator bezeichnen. Wendet man die Drehung auf ein $\psi\in\mcL^2(\R^3)$ an, gilt
        \[\mcD_{\varphi,a}\psi(x) = \psi(x) - \frac{a}{\dabs{a}{2}}\times x\cdot D_{1_\R^3}\psi(x) = \psi(x) - \frac{\cmath}{\hbar}\cdot\frac{a}{\dabs{a}}\cdot(x\times p)\psi(x),\]
        wodurch wir einen neuen Zusammenhang
        \[\mcD_{\varphi,a} = I_3 - \frac{\cmath}{\hbar}\cdot a\cdot L\]
        mit unserem definierten Drehimpuls $L$ vorfinden. 

        \begin{Aufgabe}
            \nr{} Zeige, daß $L$ ein selbstadjungierter Operator ist. 

            \nr{} Zeige durch die Symmetrie des Zentralpotentials $V:\R^3\to\R$ die Kommutatoreigenschaft $[\mcD_{\varphi,a},H] = 0$, wobei $H$ der Hamiltonoperator ist. 
        \end{Aufgabe}
        Aus der Aufgabe und der Schreibweise für $\mcD_{\varphi,a}$ folgern wir weiter $[L,H] = 0$.

        \begin{Aufgabe}
            \nr{} Beschäftige dich mit der Komponentenschreibweise der Operatoren $H$ und $L$. Zeige $[L_i,L_j]\neq 0$ für $i\neq j$. Was bedeutet hier die Notation?
        \end{Aufgabe}
        Mit der Aufgabe folgern wir nun eine allgemeinere: Für $(i,j)\in[3]^2$ gilt 
        \[[L_i,L_j] = \cmath\hbar\cdot \sum_{k=1}^3\epsilon_{ijk}\cdot L_k \Longleftrightarrow L\times L = \cmath\hbar\cdot L,\]
        wobei $\epsilon$ als Rechenhilfe unter dem Namen \textit{Levi-Civita-Symbol} bekannt ist. 

        \begin{Aufgabe}
            \nr{} Folgere nun als letzte Eigenschaft aus den bisherigen $[L^2,H] = 0$ und $[L^2,L] = 0$.
        \end{Aufgabe}
        Betrachten wir den Erwartungswert $\langle L^2\rangla_\psi$, so finden wir nach Definition 
        \[\int_\R^3\overline\psi\cdot L^2(\psi) = \sum_{i\in[3]}\int_\R^3\overline{L_i(\psi)}\cdot L_i(\psi)\geq 0.\]
        Somit sind durch Selbstadjungiertheit und positiv definit die Eigenwerte von $L^2$ im reellen und größer Null: $L^2\psi = \hbar^2\lambda\psi$. Definiere $L_\pm:=L_x\pm\cmath\cdot L_y$. Weitere Eigenschaften sind dann
        \begin{alignat*}{4}
            &\int_\R^3\psi\cdot L_\pm(\psi) &&= \int_\R^3 (L_\mp(\psi))\cdot\psi &&\qquad [L_z,L_\pm] &&= \pm\hbar\cdot L_z\\
            &[L_+,L_-] &&= 2\hbar\cdot L_z &&\qquad [L^2,L_\pm] &&= 0\\
            &L_+\circ L_- &&= L_x^2 + L_y^2 + \hbar &&\qquad L_z L^2 &&= L_-\circ L_+ + \hbar L_z + L_z^2.
        \end{alignat*}
        
        \begin{Aufgabe}
            \nr{} Zeige die Eigenschaften.
        \end{Aufgabe}
        Um den Zusammenhang zwischen $L$ und $L_\pm$ zu erkennen, behaupten wir nun $L_\pm(\psi)$ ist Eigenvektor von $L_z$ mit Eigenwert $\hbar$. Wir finden
        \[(L_z\circ L_\pm)(\psi) = (L_\pm\circ L_z)(\psi) \pm\hbar L_\pm(\psi) = h\cdot (m\pm 1)\cdot L_\pm(\psi),\]
        wobei $L_z(\psi) = h\cdot m\cdot \psi$ gelte. 

        \begin{Aufgabe}
            \nr{} Zeige $\psi$ ist Eigenvektor von $L^2$ mit Eigenwert $\hbar^2\cdot\lambda$. Zeige dann auch $L_\pm(\psi)$ ist Eigenvektor mit selbem Eigenwert. 
        \end{Aufgabe}

        Ist $\psi$ ein normierbarer Eigenvektor zu $L^2$, dann ist $L_\pm(\psi)$ ebenfalls normierbar:
        \[\int_{\R^3}\abs{L_z(\psi)}^2 = \int_{\R^3}\overline\psi\cdot(L_\mp\circ L_\pm)(\psi) = \hbar^2\cdot (\lambda-m^2\mp m)\cdot\int_{\R^3}\abs{\psi}^2 \geq 0.\]
        Wir können also sehen, daß nicht unbedingt jeder Eigenvektor von $L^2$ sich zur Normierbarkeit eignet. 

        \begin{Aufgabe}
            \nr{} Betrachte die Eigenwertgleichungen (i) $L^2\psi_{l,m} = \hbar^2\cdot l\cdot(l + 1)\cdot\psi_{l,m}$, (ii) $L_z(\psi_{l,m}) = \hbar\cdot m\cdot \psi_{l,m}$ und (iii) $L_\pm(\psi_{l,m}) = \hbar\cdot\sqrt{l\cdot (l+1) - m\cdot(m\pm 1)}\cdot\psi_{l,m}$. Was sagen die Zahlen $l,m\in\N$ aus?
        \end{Aufgabe}

    \subsection{Ortsdarstellung und Kugelflächenfunktionen}
        Wir suchen nun einen konkreten Ausdruck für die Eigenvektoren $\psi_{l,m}\in\mcL^2(\R^3)$ der Eigenwertgleichung $L^2(f) = \lambda\cdot f$. Nach der Definition von $L$ steht in ausgeschriebener Form zunächst ein monströs aussehnder Ausdruck:
        \begin{align*}
            L^2(f(x)) &= -\hbar\cdot(x\times D_h)^2(f(x)) = \hbar^2\cdot l\cdot (l+1)\cdot f(x)\\
            L_3(f(x)) &= -\cmath\hbar\cdot (x\times D_h)_3^2(f(x)) = \hbar\cdot m\cdot f(x),
        \end{align*}
        mit Identifikation $L_z$ und $L_3$. Unter der Transformation in Kugelkoordinaten $f_K$ und \emph{Trennung der Variablen} der Form $f(x) = R(\dabs{r}{2})\cdot Y(\vartheta,\varphi)$ und $Y(\vartheta,\varphi) = \Phi(\varphi)\cdot\Theta(\vartheta)$ kann man zunächst $\Phi(\varphi) = \exp(\cmath\cdot m\cdot\varphi)$ setzen. Es ist dann zu Lösen eine Gleichung der Form 
        \[\nbra{\dv{z}(1-z^2)\cdot\dv{z} + \Bbra{l\cdot(l+1) - \frac{m^2}{1-z^2}}}(\Theta(\vartheta)) = 0.\]
        Wir nennen die Gleichung auch \emph{verallgemeinerte Legendre Gleichung}, deren Lösung die \emph{verallgemeinerten Legendre Polynome} sind. Diese sind von der Form
        \[\Theta(\vartheta) = \sum_{k=0}^l a_k\cdot P_k^m(\cos(\vartheta)),\]
        wobei $P_k^m$ die \emph{asoziierten Legendre Polynome} sind. Diese sind wiederum von der Form
        \[P_k^m(z) = (-1)^m\cdot(1-z^2)^{\frac{m}{2}}\cdot\dv{z}^m\cdot P_k(z).\]
        Für $Y$ finden wir eine Gleichung der Form
        \[Y(\vartheta,\varphi) = \sqrt{\frac{2l + 1}{4\pi}\cdot\frac{(l-m)!}{(l+m)!}}\cdot P_e(\cos(\vartheta))\cdot\exp(-\cmath m\cdot\varphi),\]
        die sogenannte \emph{Kugelflächenfunktion}. 


    

    

\end{document}