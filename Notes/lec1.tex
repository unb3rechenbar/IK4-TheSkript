\documentclass{article}

\begin{document}
    \marginnote{\textbf{\textit{VL 1}}\\25.04.2023, 08:15}
    \subsection*{Einleitung}
        Bei der Auffassung kleinster Teilchen gab es Probleme mit dem Teilchenmodell. 
        \begin{Aufgabe}
            \nr{} Stelle dieses Problem \emph{deutlich} dar. Skizziere eine Lösung desselben. 
        \end{Aufgabe}
        \subsubsection*{Schwarzkörperstrahlung}
            Jede sogenannte \emph{Mode} mit der Frequenz $\nu = c_0/\lambda$ des elektromagnetischen Feldes kann beliebige Energien enthalten, enthält jedoch nach dem \emph{Äquipositionsprinzip} im Mittel die Energie $E = k_B\cdot T$, bekannt als das \emph{Rayleigh-Jeans-Gesetz}. 

        \subsubsection*{Photoeffekt}

        \subsubsection*{Compton Effekt}
            [$\to$ IK4 Exp. II] 

        \subsubsection*{Welleneigenschaften der Materie}
            [$\to$ IK4 Exp. II]

        \subsubsection*{Doppelspaltexperiment mit Elektronen}
            [$\to$ IK4 Exp. II]

        
        \begin{Aufgabe}
            \nr{} Lies im Skript der \emph{Experimentalphysik II} die Inhalte der Überschriften nach. 
            \begin{itemize}[label=$\to$]
                \item \marginnote{\scriptsize(\stift\theshw.1)}Was ist die Wellenfunktion beim Doppelspaltexperiment? Wie erklärt man, daß ein Elektron durch beide Spalten gehen kann? Was passiert mit einem einzeln eingestrahlten Elektron?
                \item \marginnote{\scriptsize(\stift\theshw.2)}Wie lautet die \emph{de Broglie Relation}?
                \item \marginnote{\scriptsize(\stift\theshw.3)}Kann man das Doppelspaltexperiment auch mit massiveren Teilchen oder Molekülen durchführen? Gibt es hierbei eine Grenze? Recherchiere den Beitrag zur Doppelspaltuntersuchung der \emph{Universität Konstanz}.
            \end{itemize}
        \end{Aufgabe}

    \subsection*{Welle-Teilchen-Dualismus}
        Wir haben beobachtet:
        \begin{itemize}[label=$\to$]
            \item elektromagnetische Wellen verhalten sich wie Teilchen
            \item materielle Teilchen verhalten sich wie Wellen
        \end{itemize}
        Als Ziel unserer folgenden Untersuchungen setzen wir eine \emph{einheitliche Theorie}, welche sowohl die Wellen- als auch die Teilcheneigenschaften beschreibt. 

    \subsection*{Wellenfunktion und Wahrscheinlichkeitsinterpretation}
        Wir wollen den folgenden Zusammenhang herstellen:
        \begin{table}[H]
            \centering
            \begin{tabular}{p{5cm}|p{6cm}}
                freies \emph{Teilchen} & ebene \emph{Welle} \\
                \hline
                Impuls $p\in\R^3$ & Wellenvektor $k\in\R^3$ \\
                Energie $E(p) = p^2/2m$ & Kreisfrequenz $\omega(k) = \hbar k^2/2m = c_0\cdot\dabs{k}{2}$ \\
                 & Amplidute am Ort $r(t)$ mit $\psi(t,r(t)) = C\cdot\exp(\cmath(\scpr{r(t)}{k} - \omega\cdot t))$ $\to$ \emph{Wellenfunktion}
            \end{tabular}
            \caption{Gegenüberstellung der Teilchen- und Welleneigenschaften.}
        \end{table}
        \noindent Es kommen nun die folgenden Fragen auf:
        \begin{itemize}[label=$\to$]
            \item Wie hängen $p$ und $k$ zusammen? 
            \item Was ist die physikalische Bedeutung von $\psi\in C^1(\R\times\R^3,\R)$?
        \end{itemize}

        \subsubsection*{Die Impuls-Wellenvektor Beziehung}
        Es stellt sich heraus, daß wir die erste Frage bereits mit der \emph{de Broglie Relation} [$\to$ IK4 Exp II] beantworten können: $p(k) = \hbar\cdot k$, wobei $\hbar:=h/(2\pi)$ mit $h = 6.6\cdot 10^{-34}\si{\joule\second}$. Für die Energie finden wir durch die Energie eines Photons den Zusammenhang $E(\omega) = h\cdot\nu = \hbar\cdot\omega$ mit $\omega = 2\pi\cdot\nu$. Dann ist $E$ mit $\nu$ als Funktion von $p$ durch die Relation $\nu = c_0 / \lambda$ und letztlich $\lambda$ als Funktion von $p$ durch die de Broglie Relation $\lambda(p) = h/p$ insgesamt die Energie - Impuls Beziehung 
        \[
            (E\circ\nu\circ\lambda)(p) = h\cdot \frac{c_0}{h / p} = c_0\cdot p =: E(p).
        \]
        Wir wollen in der angenommenen Wellenfunktion
        \[
            \psi_t(x) = C\cdot\exp(\cmath\cdot (\scpr{k}{x}-\omega\cdot t))
        \]
        Die Größen $k$ und $\omega$ durch $p$ ausdrücken. Die Transformation für $k$ ist dabei gegeben durch $k(p) = p/\hbar$, die Transformation für $\omega$ müssen wir noch durch die oben definierten Funktionen $\omega(\nu)$, $\nu(\lambda)$ und $\lambda(p) = h / p$ ausdrücken. Wir erhalten die Verkettung $(\omega\circ\nu\circ\lambda)(p) = 2\pi\cdot p\cdot c_0/h$. Nun verwenden wir die Definition von $\hbar$ und $E(p)$, sodaß wir schreiben können $(\omega\circ\nu\circ\lambda)(p) = E(p) / \hbar =: \omega(p)$. Diese beiden Ausdrücke können wir nun in die Wellenfunktion einsetzen, sodaß wir erhalten
        \[
            \psi_t(x) = C\cdot\exp(\cmath\cdot (\scpr{p/\hbar}{x}-(E(p)/\hbar)\cdot t)) = C\cdot\exp(\cmath\cdot \frac{\scpr{p}{x} - E(p)\cdot t}{\hbar}).
        \]
        Vorbereitend für eine Fouriertransformation halten wir dieses Ergebnis fest. 
        \begin{mcor}{Die ebene Wellenfunktion mit Impulsargument}
            Durch die Transformationen $\Phi:(p/\hbar)_{p\in\R}$ und $\Psi:(E(p)/\hbar)_{p\in\R}$ mit der Energiebeziehung $E:(c_0\cdot p)_{p\in\R}$ erhalten wir die ebene Wellenfunktion mit Impulsargumenten 
            \[
                \psi_t(x) = C\cdot \exp(\cmath\cdot(\scpr{\Phi(p)}{x} - \Psi(p)\cdot t)).
            \]
        \end{mcor}
        \subsubsection*{Dispersion der ebenen Welle}
        Für die Dispersion der Welle gilt 
        \[E(\omega) = \hbar\cdot\omega = \begin{cases}
            \frac{\hbar^2\cdot k^2}{2\cdot m} & m>0 \\
            \hbar\cdot c_0\cdot \dabs{k}{2} \sonst
        \end{cases} = \begin{cases}
            \frac{\scpr{p}{p}}{2\cdot m} & m>0 \\
            c_0\cdot\dabs{p}{2} \sonst
        \end{cases}\]
        % Für die physikalische Interpretation müssen wir uns der Wahrscheinlichkeitsinterpretation widmen: 
        % \begin{table}[H]
        %     \centering
        %     \begin{tabular}{p{5cm}|p{5cm}}
        %         Teilchen & Welle \\
        %         \hline
        %         Aufenthaltswahrscheinlichkeit des Teilchens (pro Volumen) am Ort $r(t)$ zur Zeit $t\in\R$ & Intensität der Welle $\abs{\psi(t,r(t))}^2$ \\
        %     \end{tabular}
        % \end{table}
        \subsubsection*{Genauigkeit der Messung}
        \noindent Prinzipiell ist es in der klassischen Physik möglich, den \emph{Ort} zum \emph{Zeitpunkt} eines Teilchens zu kennen; anders ist es bei quantenmechanischen Wellen. 
        \begin{mdef}{Aufenthaltswahrscheinlichkeit}
            Für eine quadratintegrierbare Wellenfunktion $\psi\in\mcL^2(\R)$ ist die Aufenthaltswahrscheinlichkeit definiert über das Wahrscheinlichkeitsmaß $P(\osi,\cdot)$ auf $(\R^3,\sigma(\R^3))$ als 
            \[
                P(\psi,V) := \abs{C}\cdot\int_V\abs{\psi(x)}^2\;\uplambda(dx),\qquad V\in\sigma(\R^3),
            \]
            wobei $C$ die Normierungskonstante $C = 1/P(\psi,\R)$ ist.
        \end{mdef}
        Bei einer ebenen Welle $\psi_\textit{eben}$ stellen wir sofort $P(\psi_\textit{eben},\R) = \infty$ fest. Dies zeigt uns, daß $\psi_\textit{eben}$ nur auf einem beschränkten Teilvolumen $V\subseteq\R$ normierbar ist:
        \[
            P(\psi_{eben},V) = \abs{C}^2\cdot\int 1\;\uplambda(dx) = \abs{C}^2\cdot\uplambda(V) \iff C = \frac{1}{\sqrt{\uplambda(V)}}. 
        \] 
        Hier kommen verschiedene Möglichkeiten der Normierung zum Einsatz. Prinzipiell versucht man, das betrachtete \emph{Gesamtvolumen} einzuschränken. Diesen Prozess nennt man \emph{Boxnormierung}.
        \begin{mdef}{Boxnormierung}
            Für ein echtes beschränktes Teilvolumen $V\in\sigma(\R^d)$ des $\R^d$ fordert man $P(\psi,V) = 1$ und berechnet für alle $W\subseteq V$ mit $W\in\sigma(\R^d)$ die boxnormierte Aufenthaltswahrscheinlichkeit $P_V(\psi,W) := \abs{C}^2\cdot P(\psi,W)$, wobei $C$ die aus $1/P(\psi,V)$ resultierende \emph{Normierungskonstante} ist.
        \end{mdef} 
        
        \begin{Aufgabe}
            \nr{} Überlege dir den Spezialfall eines Punktes $\nset{x}\subseteq\R^3$ als Testvolumen. Wie sieht die Aufenthaltswahrscheinlichkeit aus?
        \end{Aufgabe}

    \subsection{Wellenpakete}
        Als nächstes beschäftigen wir uns mit der Frage, wie wir Teilchen mit genau definiertem Aufenthaltsort beschreiben. Wir wenden uns hierbei an das Prinzip der \emph{Superposition}, konkreter der \emph{Fourier-Summation}, bei der wir eine Funktion $\psi\in \mcL^2(\R^3)$ zerlegen in Funktionen des Typus der ebenen Welle:
        \[
            \psi_t(x) = \frac{1}{(2\cdot\pi)^3}\cdot\int_\R(\mcF\psi_t)(k)\cdot\exp(\cmath\cdot(\scpr{k}{x} - \frac{\hbar\cdot k^2}{2\cdot m}\cdot x))\;\uplambda(dk).
        \]
        Mit unseren Zusammenhängen oben erhalten wir für die Energie den Ausdruck $E(k) = p(k)^2/(2\cdot m) = (k\cdot\hbar)^2/(2\cdot m)$, sodaß wir in die ebene Wellengleichung einsetzen können und erhalten
        \[
            \psi_t^\textit{eben}(x) = \exp(\cmath\cdot\bbra{\scpr{k}{x} - E(k)/\hbar\cdot t}) = \exp(\cmath\cdot\bbra{\scpr{k}{x} - \frac{k^2\cdot\hbar}{2\cdot m}\cdot x}).
        \]
        \begin{mdef}{Wellenpaket unter Fourier}
            Eine Funktion $\psi\in L^2(\R)$ können wir durch ebene Wellenfunktionen $\psi^\textit{eben}_t(x)$ zerlegen, wobei die Vorfaktorenfunktion $C:k\mapsto (\mcF\psi_t)(k)$ im Wellenvektorraum gegeben ist. Damit gilt dann
            \[
                \psi_t(x) = \frac{1}{(2\cdot\pi)^3}\cdot\int_\R(\mcF\psi_t)(k)\cdot\psi^{eben}_t(x)\;\uplambda(dk).
            \]
        \end{mdef}

        \begin{Aufgabe}
            \nr{} Warum wird bei der Fourier-Summation keine Wurzel im Vorfaktor gezogen? Recherchiere verschiedene Konventionen. [\textit{Tipp:} Bedenke $\hbar = h/(2\cdot \pi)$ und die Definition des Impulses über $k$.]
        \end{Aufgabe}
\end{document}