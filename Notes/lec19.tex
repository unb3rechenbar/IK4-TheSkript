\documentclass{subfiles}

\begin{document}
    \marginnote{\textbf{\textit{VL 19}}\\03.07.2023, 11:45}
    \subsection{Das Eigenwertproblem}
        Bei der Schrödingergleichung handelt es sich um ein Eigenwertproblem der Form $H(\psi) = \lambda\cdot\psi$ mit $H\in L_S(\mcH)$, $\psi\in\mcH$ und $\lambda\in\C$ für ein komplexen Hilbertraum $\mcH$. Es heißt dann $\lambda$ Eigenwert und $\psi$ Eigenvektor zum Eigenwert $\lambda$. Gibt es mehrere Eigenvektoren zu einem Eigenwert, so bilden diese einen Unterraum, den Eigenraum zum Eigenwert $\lambda$. Wir bezeichnen dann $\lambda$ als \emph{$n$-fach entartet}, falls $\dim E_\lambda = n$. Die Sammlung aller Eigenwerte ist das sogenannte \emph{Punktspektrum} $\sigma_P(H)$ von $H$.
        \begin{Aufgabe}
            \nr{} Überlege dir jeweils Beispiele zu einem (i) diskreten Punktspektrum, (ii) kontinuierlichen Punktspektrum oder (iii) gemischten Punktspektrum. 

            \nr{} Betrachte noch einmal den Spektralsatz selbstaudjungierter linearer Operatoren. Formuliere ihn einmal für \emph{normale} $H\in L_S(\mcH)$ und notiere die Beziehung zur selbstadjungierten Formulierung. 

            \nr{} Zeige die Existenz einer ONB aus Eigenvektoren für selbstadjungierte Operatoren mithilfe des Spektralsatzes der Funktionalanalysis. 

            \nr{} Zeige für eine ONB $\varphi\in\mcH^I$ die Eigenwertgleichungsbeziehung $A(\varphi_i) = \lambda_i\cdot \varphi_i$ für $A\in L_S(\mcH)$ und $\lambda_i\in\C$. Welche Forderung stellt sich an die Indexmenge $I$?

            \nr{} Berechne $q_n(A)$ als Funktionsauswertung mit $q$ als $n$-te Potenzfunktion. 

            \nr{} Nutze den Einsetzungshomomorphismus um das Polynom $p(A)$ $n$-ten Grades zu beschreiben.
        \end{Aufgabe}

    \subsection{Messprozess in der Quantenmechanik}
        Wir wollen uns nun mit dem fundamentalen Standbein der Quantenmechanik beschäftigen. Für ein vorliegendes quantenmechanisches, also mikroskopisches System gilt die \emph{Messung} als Zusammenhang zu dem makroskopischen, nicht quantenmechanischen Beobachter. 
        
        Im klassischen System ist die Auswirkung des Beobachters durch eine Messung, also eine Störung des Systemablaufes, im Prinzip beliebig klein. In der Quantenmechanik ist dies jedoch durch die Unschärfenrelation nach unten beschränkt. Eine weiter Besonderheit ist der Verlust von Informationen nach der Messung; Misst man beispielsweise den Ort, so kann man nicht erneut den Impuls messen, da die Wellenfunktion nach der Messung nicht mehr die ursprüngliche ist. Wir wollen uns nun mit dem Ablauf einer quantenmechanischen Messung beschäftigen. 
        
\end{document}