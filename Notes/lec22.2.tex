\documentclass{subfiles}

\begin{document}
Bisher haben wir uns ausschließlich der zeitunabhängigen Schrödingergleichung $H(\psi) = \lambda\cdot\psi$ gewidmet. Bisherige Beispiele waren dabei immer \emph{analytisch lösbar} (z.B. harm. Osz., Wasserstoffatom, Potentialtöpfe). Dies ist für die meisten Potentiale $V$ jedoch nicht mehr möglich (z.B. Wasserstoffmolekül, anharmonischer Oszillator, Wasserstoffatom im el. Feld). Hier verwendet man die \emph{Störungstheorie} zur approximativen Lösung. Es gibt dabei verschiedene Kategorien der Lösung:
\begin{itemize}[label=$\to$]
    \item \emph{Variationsmethoden}. Prominente Beispiele sind hier die \emph{Variationsmethode von Rayleigh} und die \emph{Variationsmethode von Hylleraas}, bzw. das \emph{Variational Quantum Eigensolver (VQE)}-Verfahren.
    \item \emph{Quasiklassische Näherung}. Ein Beispiel wäre hier die \textit{Wagner-Kramers-Brillouin-Näherung}.
    \item \emph{Störungstheorie}. Hier werden wir im weiteren ansetzen. 
\end{itemize}
\begin{Aufgabe}
    \nr{} Nähere $\sqrt{26}$ mithilfe von Taylorentwicklung an. 
\end{Aufgabe}
\end{document}