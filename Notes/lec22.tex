\documentclass{subfiles}

\begin{document}
    \marginnote{\textbf{\textit{VL 22}}\\12.07.2023, 11:45}
    \subsubsection*{Eigenschaften des Zähloperators N}
        \begin{Aufgabe}
            \nr{} Berechne $\dabs{N}{} = 0$,$\dabs{a(x)}{}^2 = \lambda$ und $\dabs{a^*(x)}{}^2 = \lambda$. 

            \nr{} Berechne die Auswertung $a(\ket{n})$ für $\ket{n}\in\Def(a)$ und $a^*(\ket{n})$ für $\ket{n}\in\Def(a^*)$. Bestimme dadurch die Eigenwertfolgen. 
        \end{Aufgabe}

        Mit der Aufgabe erkennen wir dann die Operatorauswertungsfolgen
        \begin{align*}
            \fdef{a(\ket{n})}{n\in\N} &= \fdef{\begin{cases}
                0 & n=0 \\
                1/\sqrt{n} \cdot \ket{n-1} & n\in\N
            \end{cases}}{n\in\N_0}, \\
            \fdef{a^*(\ket{n})}{n\in\N} &= \fdef{\begin{cases}
                1/\sqrt{n+1} \cdot \ket{n+1} & n\in\N \\
                0 & n=0
            \end{cases}}{n\in\N_0}.
        \end{align*}
        Der Vorteil der Wahl von $N$ als Ausdruck in $H = \hbar\omega\cdot(N + 1/2)$ ist nun der optisch identische Ausdruck der zugehörigen Eigenwertfolge $E(n) = \hbar\omega\cdot (n+1)$ für $n\in\N_0$. 

    \section{Störungstheorie}
        Bisher haben wir uns ausschließlich der zeitunabhängigen Schrödingergleichung $H(\psi) = \lambda\cdot\psi$ gewidmet. Bisherite Beispiele waren dabei immer \emph{analytisch lösbar} (Beispiel harm. Osz., Wasserstoffatom, Potentialtöpfe). Dies ist für die meisten Potentiale $V$ jedoch nicht mehr möglich (z.B. Wasserstoffmolekül, anharmonischer Oszillator, Wasserstoffatom im el. Feld). Hier verwendet man die \emph{Störungstheorie} zur approximativen Lösung. Es gibt dabei verschiedene Kategorien der Lösung:
        \begin{itemize}[$\to$]
            \item \emph{Variationsmethoden}. Prominente Beispiele sind hier die \emph{Variationsmethode von Rayleigh} und die \emph{Variationsmethode von Hylleraas}, bzw. das \emph{Variation Quantum Eigensolver (VQE)}-Verfahren.
            \item \emph{Quasiklassische Näherung}. Ein Beispiel wäre hier die \textit{Wagner-Kramers-Brillouin-Näherung}.
            \item \emph{Störungstheorie}. Hier werden wir im weiteren ansetzen. 
        \end{itemize}
        \begin{Aufgabe}
            \nr{} Nähere $\sqrt{26}$ mithilfe von Taylorentwicklung an. 
        \end{Aufgabe}

\end{document}