\documentclass{article}

\begin{document}
    \subsection*{Einleitung}
        Bei der Auffassung kleinster Teilchen gab es Probleme mit dem Teilchenmodell. 
        \begin{Aufgabe}
            \nr{} Stelle dieses Problem \emph{deutlich} dar. Skizziere eine Lösung desselben. 
        \end{Aufgabe}
        \subsubsection*{Schwarzkörperstrahlung}
            Jede sogenannte\emph{Mode} mit der Frequenz $\nu = c_0/\lambda$ des elektromagnetischen Feldes kann beliebige Energien enthalten, enthält jedoch nach dem \emph{Äquipositionsprinzip} im Mittel die Energie $E = k_B\cdot T$, bekannt als das \emph{Rayleigh-Jeans-Gesetz}. 

        \subsubsection*{Photoeffekt}


        \subsubsection*{Compton Effekt}
            [$\to$ IK4 Exp. II] 

        \subsubsection*{Welleneigenschaften der Materie}
            [$\to$ IK4 Exp. II]

\end{document}